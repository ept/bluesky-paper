\documentclass[sigconf,review]{acmart}
\setcopyright{none}

\begin{document}
\title{Bluesky and the AT Protocol:\\ A Foundation for Usable Decentralized Social Media}
\author{Martin Kleppmann}
\email{martin@kleppmann.com}
\orcid{0000-0001-7252-6958}
\affiliation{%
  \institution{TU Munich}
  \city{Munich}
  \country{Germany}
}

\author{TODO: review author list}
\author{Paul Frazee, Jake Gold, Jay Graber}
\author{Daniel Holmgren, Devin Ivy, Jeromy Johnson}
\author{Bryan Newbold, Jaz Volpert}
\affiliation{%
  \institution{Bluesky PBC}
  \country{United States}
}

\begin{abstract}
TODO
\end{abstract}

\begin{CCSXML}
<ccs2012>
   <concept>
       <concept_id>10002951.10003260.10003282.10003292</concept_id>
       <concept_desc>Information systems~Social networks</concept_desc>
       <concept_significance>500</concept_significance>
   </concept>
   <concept>
       <concept_id>10003033.10003106.10003114.10003118</concept_id>
       <concept_desc>Networks~Social media networks</concept_desc>
       <concept_significance>500</concept_significance>
   </concept>
   <concept>
       <concept_id>10003456.10003457.10003490.10003507.10003508</concept_id>
       <concept_desc>Social and professional topics~Centralization / decentralization</concept_desc>
       <concept_significance>300</concept_significance>
   </concept>
 </ccs2012>
\end{CCSXML}

\ccsdesc[500]{Information systems~Social networks}
\ccsdesc[500]{Networks~Social media networks}
\ccsdesc[300]{Social and professional topics~Centralization / decentralization}

\keywords{social media, decentralization, federation, web architecture}
\maketitle

\section{Introduction}

Over the last two decades, social media services have evolved from a fun curiosity into a cornerstone of civic life~\cite{Barabas:2017}.
This development has been accompanied by increasing unease that mainstream ``digital town squares'', such as Twitter/X or Facebook, are under the control of a single corporation, and may change their policies on the whim of their leaders~\cite{Yeung:2023}.
Their operations are opaque (e.g.\ regarding which content is recommended to users), and their users lack agency over their user experience.
As a result, there has been increasing interest in decentralized social networks, of which the \emph{fediverse} around the ActivityPub protocol~\cite{ActivityPub} and the Mastodon software~\cite{Mastodon} is perhaps the best known (we review a selection of decentralized social networks in Section~\ref{sec:related-work}).

However, decentralization also introduces new challenges.
For example, in the case of Mastodon, a user needs to choose a server when creating an account.
This choice is significant because the server name becomes part of the username; migrating to another server implies changing username, and preserving one's followers during such a migration requires the cooperation of the old server.
If a server is shut down without warning, accounts on that server cannot be recovered~-- a particular risk with volunteer-run servers.
In principle, a user can host their own server, but only a small fraction of social media users have both the technical skills and the inclination to do so.

The distinction between servers in Mastodon introduces complexity for users that does not exist in centralized services.
For example, a user viewing a thread of replies in the web interface of one server may see a different set of replies compared to viewing the same thread on another server, because a server only shows those replies that it knows about~\cite{Adida:2022}.
As another example, when viewing the web profile of an account on another server, clicking the ``follow'' button does not simply follow that account; instead, the user needs to enter the hostname of their own server and be redirected to a URL on their home server before they can follow the account.
In our opinion, it is undesirable to burden users with such complexity arising from the federated architecture.

In this paper we introduce the \emph{AT Protocol} (atproto), a decentralized foundation for social networking, and \emph{Bluesky}, a Twitter-style social app built upon it.
A core design goal of atproto and Bluesky is to enable a user experience of the same or better quality as centralized services, while being open and decentralized on a technical level.
We introduce the user-facing features of Bluesky in Section~\ref{sec:product}, and in Section~\ref{sec:architecture} we explain the underlying systems architecture.
The AT Protocol is designed such that for every part of the system there are multiple competing operators providing interoperable services, making it easy to switch from one provider to another.

Decentralization alone is not able to solve some of the thorniest problems of social media, such as misinformation, harassment, and hate speech.
However, by opening up the internals of a service to contributors who are not employees of a particular company, decentralization can enable a marketplace of approaches to these problems.
For example, Bluesky allows anybody to create custom feeds that can apply arbitrary filtering and content selection methods, and custom client software can access all features of the ``official'' app, including moderation tools.
Our hope is that this architectural openness will enable communities to develop their own approaches to managing problematic content, without depending on a centralized service operator to implement these features for them.

For example, researchers wanting to identify disinformation campaigns can easily get access to all content being posted, the social graph, and user profiles on Bluesky.
If they are able construct an algorithm to label suspected disinformation, they can publish their labels in real time for use by other parts of the system, such as custom feed generators or client apps.
One goal of this paper is therefore to bring Bluesky and the AT Protocol to the attention of researchers working on such algorithms, and to invite them to use the rapidly growing dataset of Bluesky content as a basis for their work.

\section{The Bluesky Social App}\label{sec:product}

Bluesky presents itself to users as a straightforward microblogging application in the style of Twitter/X (see Figure~\ref{fig:home-feed}).
The ``official'' client app is available on iOS, Android, and the web.
Users can make public posts containing up to 300 characters of text, and up to four images, and they can interact with posts by replying, reposting, or liking.
A user can also follow other users, and the default feed shows posts by accounts that the user is following in reverse chronological order.
There are also alternative feeds that show content on various topics, without the user needing to follow the poster (see Section~\ref{sec:feeds}), which helps users discover each other.

\begin{figure}
    \centering
    \includegraphics[width=0.7\linewidth]{home-feed.png}
    \Description{A Twitter-like feed of short posts. At the top is a feed selector, in which the default ``Following'' feed is active.}
    \caption{Screenshot of the Bluesky home screen.}
    \label{fig:home-feed}
\end{figure}

\begin{figure}
    \centering
    \includegraphics[width=\linewidth]{user-growth.pdf}
    \Description{An exponential growth curve, approximately doubling every month, starting at 55k in May 2023 and exceeding 1.3M in October 2023.}
    \caption{Number of registered Bluesky users from May to October 2023.}
    \label{fig:user-growth}
\end{figure}

Bluesky launched an invite-only beta release in February 2023, and has grown to over 1.5~million registered users in October 2023, as shown in Figure~\ref{fig:user-growth}.
At the time of writing, an invitation code is required to create an account, and codes are available through a waitlist or from an existing user.
Such control over user signups may seem contradictory for a supposedly open and decentralized network, but it has been necessary to limit the load on our infrastructure and to keep abuse at a manageable level.
We intend to remove the need for invite codes in early 2024.

Bluesky, PBC (a public-benefit corporation) develops the official client app and operates the core services; the client and several server-side components are open source under the MIT license~\cite{BlueskyGithub}.
The protocols they use are defined by open specifications~\cite{AtProtoSpecs}.
Several parts of the system, such as feed generators (Section~\ref{sec:feeds}) and various alternative clients~\cite{AtProtoClients} are developed and operated by independent third parties.

\subsection{Moderation Features}\label{sec:moderation}

Bluesky has several moderation mechanisms for managing unwanted content:
\begin{description}
    \item[Content filtering:] Automated systems label potentially problematic content (such as images of a sexual or violent nature, posts promoting hate groups, or spam), and the app's preferences allow users to choose whether to show or hide content in each of these categories in their feeds.
    \item[Mute:] A user can mute specific accounts or threads, which hides the muted content from their own feeds and notifications. The content continues to be visible to other users, and the target does not know that they were muted. A user can also publish a mutelist of accounts, and other users can subscribe to that list, which has the same effect as if they individually muted all of the accounts on the list.
    \item[Block:] One user can block another, which prevents all future interactions (such as mentions, replies, or reposts) between those accounts in addition to muting.
    \item[Takedown:] Users can report content that violates the terms of service to server operators, and the operators can take down violating posts or accounts.
    \item[Custom feeds:] While the aforementioned mechanisms provide negative moderation (helping users avoid content they do not want to see), feed generators (see Section~\ref{sec:feeds}) can actively select high-quality content.
\end{description}

Additional moderation mechanisms are under discussion~\cite{Moderation}.

\subsection{User Handles}\label{sec:handles}

\subsection{Custom Feeds}\label{sec:feeds}

\cite{AlgorithmicChoice}

\section{The AT Protocol Architecture}\label{sec:architecture}

Almost all data created by Bluesky users is public; in particular, this includes user profiles, posts, follows, and likes.
Blocking actions are also public; however, we are investigating mechanisms for making blocking information private~\cite{PrivateBlocks}.
At present, Bluesky does not support direct messages, which would need to be private.
Only a small amount of user state is currently private: any muted accounts and threads, and the unread status of notifications.

% Labeling services https://github.com/bluesky-social/proposals/tree/main/0002-labeling-and-moderation-controls

% Federation architecture https://blueskyweb.xyz/blog/5-5-2023-federation-architecture

\section{Related Work}\label{sec:related-work}

% https://activitypub.rocks/

% https://scuttlebutt.nz/
% https://github.com/lens-protocol/core
% https://nostr.com/
% "the Nostr protocol powers Minds and Snort, among other decentralized social media"

% https://www.farcaster.xyz/
% https://github.com/farcasterxyz/protocol/blob/main/docs/OVERVIEW.md
% https://www.varunsrinivasan.com/2022/01/11/sufficient-decentralization-for-social-networks

% for likes, the record is around 100 bytes, and the the MST tree stuff is anywhere from 1-4kb (depending on the repo and other randomness in the MST), usually averaging around 2.5Kb.
% https://github.com/bluesky-social/discuss/discussions/16#discussioncomment-7233086

\begin{acks}
TODO
\end{acks}

\bibliographystyle{ACM-Reference-Format}
\bibliography{references}
\end{document}
